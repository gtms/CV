% ------------------------------------
% Dario Taraborelli
% Typesetting your academic CV in LaTeX
%
% URL: http://nitens.org/taraborelli/cvtex
% DISCLAIMER: This template is provided for free and without any guarantee
% that it will correctly compile on your system if you have a non-standard
% configuration.
% Some rights reserved: http://creativecommons.org/licenses/by-sa/3.0/
% ------------------------------------

% !TEX TS-program = xelatex
% !TEX encoding = UTF-8 Unicode

\documentclass[11pt, a4paper]{article}
\usepackage{fontspec}

% BIBLIOGRAPHY MANAGEMENT
\usepackage{natbib}
\usepackage{bibentry}
\nobibliography*

% DOCUMENT LAYOUT
\usepackage{geometry}
\geometry{a4paper, textwidth=5.5in, textheight=8.5in, marginparsep=7pt, marginparwidth=.6in}
\setlength\parindent{0in}

% FONTS
\usepackage{xunicode} \usepackage{xltxtra}
\defaultfontfeatures{Mapping=tex-text} % converts LaTeX specials (``quotes'' ---
                                       % dashes etc.) to unicode
\setromanfont [Ligatures={Common}, Numbers={OldStyle}]{Minion Pro}
\setmonofont[Scale=MatchLowercase]{Consolas}
\setsansfont[Scale=MatchLowercase]{Myriad Pro}
% ---- CUSTOM AMPERSAND
\newcommand{\amper}{{\fontspec[Scale=.95]{Minion Pro}\selectfont\itshape\&}}
% ---- MARGIN YEARS
\usepackage{marginnote}
\newcommand{\years}[1]{\marginnote{\scriptsize #1}}
\renewcommand*{\raggedleftmarginnote}{}
\setlength{\marginparsep}{7pt}
\reversemarginpar

% HEADINGS
\usepackage{sectsty}
\usepackage[normalem]{ulem}
\sectionfont{\sffamily\mdseries\large\underline}
% \sectionfont{\sffamily\mdseries\large}
\subsectionfont{\rmfamily\mdseries\scshape\normalsize}
\subsubsectionfont{\rmfamily\bfseries\upshape\normalsize}

% PDF SETUP
% ---- FILL IN HERE THE DOC TITLE AND AUTHOR
\usepackage[xetex, bookmarks, colorlinks, breaklinks,
pdftitle={GilTomas-vita},pdfauthor={Gil Tomás}]{hyperref}
\hypersetup{linkcolor=blue,citecolor=blue,filecolor=black,urlcolor=blue}

% DOCUMENT
\begin{document}
\textsf{\LARGE Gil Tomás}\\[1cm]
Calvijnstraat 24\\
\texttt{1000} Brussels\\
\textsc{belgium}\\[.2cm]
Phone: \texttt{+32(0)487-371731}\\
% Fax: \texttt{609-924-8399}\\[.2cm]
email: \href{mailto:gil.tms@gmail.com}{gil.tms@gmail.com}\\
% \textsc{url}: \href{http://www.ias.edu/spfeatures/einstein/}{http://www.ias.edu/spfeatures/einstein/}\\
\vfill
Born:  September 9, 1977---Paray-le-Monial, France\\
Nationality:  Portuguese

%% \hrule
\section*{Current position}
PhD candidate, \textsc{iribhm}, Université Libre de
Bruxelles

%% \hrule
\section*{Areas of specialization}
Molecular Biology; Oncology; Bioinformatics.

%% \hrule
\section*{Work Experience}

\noindent\years{2002-2004}Researcher, \textsc{ipatimup}, Porto, Portugal\\
\years{2003-2004}Researcher, Univerity of Maryland, \textsc{usa}\\
\years{2005-2007}Self-employed, translator, private tutor (Biology, Chemestry, Mathematics)\\
\years{2008-2009}Researcher, \textsc{iribhm}, Université Libre de Bruxelles, Belgium\\
\years{2009-2014}PhD Candidate, \textsc{iribhm}, Université Libre de Bruxelles, Belgium\\

% \hrule
\section*{Education}

\noindent\years{2001}\textsc{MSc} in Biology, Universidade do Porto \texttt{(15/20)}\\

% \hrule
% \section*{Grants, honors \amper{} awards}
% \noindent\years{1921}Nobel Prize in Physics, Nobel Foundation

\section*{Personal Skills \amper{} Competences}
\subsection*{Language Skills}
Fluent in English, French and Portuguese.  Proficient in Dutch.

\subsection*{Programming Competences}
Fluent in \textsf{Unix} systems, \textsf{Emacs}, \textsf{Git}, \textsf{R}
programming language and Bioconductor.

Experience in managing complex software development projects.

\section*{Publications}

\subsection*{Journal articles}
\noindent\years{2002}\bibentry{tomas_peopling_2002} \texttt{(cited by 29)}

\noindent\years{2012\emph{a}}\bibentry{tomas_general_2012} \texttt{(cited by 6)}

\noindent\years{2012\emph{b}}\bibentry{van_staveren_role_2012} \texttt{(cited by 3)}

\noindent\years{2013\emph{a}}\bibentry{dom_5-aza-2-deoxycytidine_2013} \texttt{(cited by 3)}

\noindent\years{2013\emph{b}}\bibentry{gomez_patterns_2013} \texttt{(cited by 3)}
% \noindent\years{1901}Einstein, Albert (1901), “Folgerungen aus den Capillaritätserscheinungen (Conclusions Drawn from the Phenomena of Capillarity)", \emph{Annalen der Physik} 4: 513\\
% \years{1905a}Einstein, Albert (1905), “On a Heuristic Viewpoint Concerning the Production and Transformation of Light", \emph{Annalen der Physik} 17: 132–148.\\
% \years{1905b}Einstein, Albert (1905), A new determination of molecular dimensions. \emph{PhD dissertation}.\\
% \years{1905c}Einstein, Albert (1905), “On the Motion—Required by the Molecular Kinetic Theory of Heat—of Small Particles Suspended in a Stationary Liquid", \emph{Annalen der Physik} 17: 549–560.
% \years{1905d}Einstein, Albert (1905), “On the Electrodynamics of Moving Bodies", \emph{Annalen der Physik} 17: 891–921.\\
% \years{1905e}Einstein, Albert (1905), “Does the Inertia of a Body Depend Upon Its Energy Content?", \emph{Annalen der Physik} 18: 639–641.\\
% \years{1915}Einstein, Albert (1915), “Die Feldgleichungen der Gravitation (The Field Equations of Gravitation)", \emph{Koniglich Preussische Akademie der Wissenschaften}: 844–847\\
% \years{1917a}Einstein, Albert (1917), “Kosmologische Betrachtungen zur allgemeinen Relativitätstheorie (Cosmological Considerations in the General Theory of Relativity)", \emph{Koniglich Preussische Akademie der Wissenschaften}\\
% \years{1917b}Einstein, Albert (1917), “Zur Quantentheorie der Strahlung (On the Quantum Mechanics of Radiation)", \emph{Physikalische Zeitschrift} 18: 121–128

\subsection*{Translated Books}

\noindent\years{2005} ``See No Evil: The True Story of a Ground Soldier
in the CIA's War on Terrorism,'' Robert Baer, \emph{Quasi Edições}, 2005

\noindent\years{2006\emph{a}} ``Intelligence: A Very Short Introduction,'' Ian
J. Dary, \emph{Quasi Edições}, 2006

\noindent\years{2006\emph{b}} ``Dreaming: An Introduction to the Science of
Sleep,'' J. Alan Hobson, \emph{Quasi Edições}, 2006


\noindent\years{2007\emph{a}} ``An Introduction to Christianity (Introduction to
Religion),'' Linda Woodhead, \emph{Quasi Edições}, 2007


\noindent\years{2007\emph{b}} ``The Good German,'' Joseph Canon, \emph{Quasi Edições}, 2007


% \noindent\years{1954}Einstein, Albert (1954), \emph{Ideas and Opinions}, New York: Random House, ISBN 0-517-00393-7

% \subsection*{Newspaper articles}

% \noindent\years{1940}Einstein, Albert, et al. (December 4, 1948), “To the editors", \emph{New York Times}\\
% \years{1949}Einstein, Albert (May 1949), “Why Socialism?", \emph{Monthly Review}.

% \section*{Teaching}

% ...

% % \hrule
% \section*{Service to the profession}

% ...

\bibliographystyle{ieeetr}
\nobibliography{cv.bib}

% \vspace{1cm}
\vfill{}
% \hrulefill

\begin{center}
  {\scriptsize  Last updated: \today\- •\- Typeset in \href{http://nitens.org/taraborelli/cvtex}{
      \fontspec{Times New Roman}\XeTeX }}\\
  % ---- FILL IN THE FULL URL TO YOUR CV HERE
  % \href{http://nitens.org/taraborelli/cvtex}{http://nitens.org/taraborelli/cvtex}}
\end{center}

\end{document}
